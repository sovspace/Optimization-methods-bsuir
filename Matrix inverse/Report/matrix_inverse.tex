\documentclass[11pt]{article}

    \usepackage[breakable]{tcolorbox}
    \usepackage{parskip} % Stop auto-indenting (to mimic markdown behaviour)
    
	\usepackage[T2A]{fontenc}
	\usepackage[utf8]{inputenc}
	\usepackage[russian]{babel}

    % Basic figure setup, for now with no caption control since it's done
    % automatically by Pandoc (which extracts ![](path) syntax from Markdown).
    \usepackage{graphicx}
    % Maintain compatibility with old templates. Remove in nbconvert 6.0
    \let\Oldincludegraphics\includegraphics
    % Ensure that by default, figures have no caption (until we provide a
    % proper Figure object with a Caption API and a way to capture that
    % in the conversion process - todo).
    \usepackage{caption}
    \DeclareCaptionFormat{nocaption}{}
    \captionsetup{format=nocaption,aboveskip=0pt,belowskip=0pt}

    \usepackage{float}
    \floatplacement{figure}{H} % forces figures to be placed at the correct location
    \usepackage{xcolor} % Allow colors to be defined
    \usepackage{enumerate} % Needed for markdown enumerations to work
    \usepackage{geometry} % Used to adjust the document margins
    \usepackage{amsmath} % Equations
    \usepackage{amssymb} % Equations
    \usepackage{textcomp} % defines textquotesingle
    % Hack from http://tex.stackexchange.com/a/47451/13684:
    \AtBeginDocument{%
        \def\PYZsq{\textquotesingle}% Upright quotes in Pygmentized code
    }
    \usepackage{upquote} % Upright quotes for verbatim code
    \usepackage{eurosym} % defines \euro
    \usepackage[mathletters]{ucs} % Extended unicode (utf-8) support
    \usepackage{fancyvrb} % verbatim replacement that allows latex
    \usepackage{grffile} % extends the file name processing of package graphics 
                         % to support a larger range
    \makeatletter % fix for old versions of grffile with XeLaTeX
    \@ifpackagelater{grffile}{2019/11/01}
    {
      % Do nothing on new versions
    }
    {
      \def\Gread@@xetex#1{%
        \IfFileExists{"\Gin@base".bb}%
        {\Gread@eps{\Gin@base.bb}}%
        {\Gread@@xetex@aux#1}%
      }
    }
    \makeatother
    \usepackage[Export]{adjustbox} % Used to constrain images to a maximum size
    \adjustboxset{max size={0.9\linewidth}{0.9\paperheight}}

    % The hyperref package gives us a pdf with properly built
    % internal navigation ('pdf bookmarks' for the table of contents,
    % internal cross-reference links, web links for URLs, etc.)
    \usepackage{hyperref}
    % The default LaTeX title has an obnoxious amount of whitespace. By default,
    % titling removes some of it. It also provides customization options.
    \usepackage{titling}
    \usepackage{longtable} % longtable support required by pandoc >1.10
    \usepackage{booktabs}  % table support for pandoc > 1.12.2
    \usepackage[inline]{enumitem} % IRkernel/repr support (it uses the enumerate* environment)
    \usepackage[normalem]{ulem} % ulem is needed to support strikethroughs (\sout)
                                % normalem makes italics be italics, not underlines
    \usepackage{mathrsfs}
    



    
    % Colors for the hyperref package
    \definecolor{urlcolor}{rgb}{0,.145,.698}
    \definecolor{linkcolor}{rgb}{.71,0.21,0.01}
    \definecolor{citecolor}{rgb}{.12,.54,.11}

    % ANSI colors
    \definecolor{ansi-black}{HTML}{3E424D}
    \definecolor{ansi-black-intense}{HTML}{282C36}
    \definecolor{ansi-red}{HTML}{E75C58}
    \definecolor{ansi-red-intense}{HTML}{B22B31}
    \definecolor{ansi-green}{HTML}{00A250}
    \definecolor{ansi-green-intense}{HTML}{007427}
    \definecolor{ansi-yellow}{HTML}{DDB62B}
    \definecolor{ansi-yellow-intense}{HTML}{B27D12}
    \definecolor{ansi-blue}{HTML}{208FFB}
    \definecolor{ansi-blue-intense}{HTML}{0065CA}
    \definecolor{ansi-magenta}{HTML}{D160C4}
    \definecolor{ansi-magenta-intense}{HTML}{A03196}
    \definecolor{ansi-cyan}{HTML}{60C6C8}
    \definecolor{ansi-cyan-intense}{HTML}{258F8F}
    \definecolor{ansi-white}{HTML}{C5C1B4}
    \definecolor{ansi-white-intense}{HTML}{A1A6B2}
    \definecolor{ansi-default-inverse-fg}{HTML}{FFFFFF}
    \definecolor{ansi-default-inverse-bg}{HTML}{000000}

    % common color for the border for error outputs.
    \definecolor{outerrorbackground}{HTML}{FFDFDF}

    % commands and environments needed by pandoc snippets
    % extracted from the output of `pandoc -s`
    \providecommand{\tightlist}{%
      \setlength{\itemsep}{0pt}\setlength{\parskip}{0pt}}
    \DefineVerbatimEnvironment{Highlighting}{Verbatim}{commandchars=\\\{\}}
    % Add ',fontsize=\small' for more characters per line
    \newenvironment{Shaded}{}{}
    \newcommand{\KeywordTok}[1]{\textcolor[rgb]{0.00,0.44,0.13}{\textbf{{#1}}}}
    \newcommand{\DataTypeTok}[1]{\textcolor[rgb]{0.56,0.13,0.00}{{#1}}}
    \newcommand{\DecValTok}[1]{\textcolor[rgb]{0.25,0.63,0.44}{{#1}}}
    \newcommand{\BaseNTok}[1]{\textcolor[rgb]{0.25,0.63,0.44}{{#1}}}
    \newcommand{\FloatTok}[1]{\textcolor[rgb]{0.25,0.63,0.44}{{#1}}}
    \newcommand{\CharTok}[1]{\textcolor[rgb]{0.25,0.44,0.63}{{#1}}}
    \newcommand{\StringTok}[1]{\textcolor[rgb]{0.25,0.44,0.63}{{#1}}}
    \newcommand{\CommentTok}[1]{\textcolor[rgb]{0.38,0.63,0.69}{\textit{{#1}}}}
    \newcommand{\OtherTok}[1]{\textcolor[rgb]{0.00,0.44,0.13}{{#1}}}
    \newcommand{\AlertTok}[1]{\textcolor[rgb]{1.00,0.00,0.00}{\textbf{{#1}}}}
    \newcommand{\FunctionTok}[1]{\textcolor[rgb]{0.02,0.16,0.49}{{#1}}}
    \newcommand{\RegionMarkerTok}[1]{{#1}}
    \newcommand{\ErrorTok}[1]{\textcolor[rgb]{1.00,0.00,0.00}{\textbf{{#1}}}}
    \newcommand{\NormalTok}[1]{{#1}}
    
    % Additional commands for more recent versions of Pandoc
    \newcommand{\ConstantTok}[1]{\textcolor[rgb]{0.53,0.00,0.00}{{#1}}}
    \newcommand{\SpecialCharTok}[1]{\textcolor[rgb]{0.25,0.44,0.63}{{#1}}}
    \newcommand{\VerbatimStringTok}[1]{\textcolor[rgb]{0.25,0.44,0.63}{{#1}}}
    \newcommand{\SpecialStringTok}[1]{\textcolor[rgb]{0.73,0.40,0.53}{{#1}}}
    \newcommand{\ImportTok}[1]{{#1}}
    \newcommand{\DocumentationTok}[1]{\textcolor[rgb]{0.73,0.13,0.13}{\textit{{#1}}}}
    \newcommand{\AnnotationTok}[1]{\textcolor[rgb]{0.38,0.63,0.69}{\textbf{\textit{{#1}}}}}
    \newcommand{\CommentVarTok}[1]{\textcolor[rgb]{0.38,0.63,0.69}{\textbf{\textit{{#1}}}}}
    \newcommand{\VariableTok}[1]{\textcolor[rgb]{0.10,0.09,0.49}{{#1}}}
    \newcommand{\ControlFlowTok}[1]{\textcolor[rgb]{0.00,0.44,0.13}{\textbf{{#1}}}}
    \newcommand{\OperatorTok}[1]{\textcolor[rgb]{0.40,0.40,0.40}{{#1}}}
    \newcommand{\BuiltInTok}[1]{{#1}}
    \newcommand{\ExtensionTok}[1]{{#1}}
    \newcommand{\PreprocessorTok}[1]{\textcolor[rgb]{0.74,0.48,0.00}{{#1}}}
    \newcommand{\AttributeTok}[1]{\textcolor[rgb]{0.49,0.56,0.16}{{#1}}}
    \newcommand{\InformationTok}[1]{\textcolor[rgb]{0.38,0.63,0.69}{\textbf{\textit{{#1}}}}}
    \newcommand{\WarningTok}[1]{\textcolor[rgb]{0.38,0.63,0.69}{\textbf{\textit{{#1}}}}}
    
    
    % Define a nice break command that doesn't care if a line doesn't already
    % exist.
    \def\br{\hspace*{\fill} \\* }
    % Math Jax compatibility definitions
    \def\gt{>}
    \def\lt{<}
    \let\Oldtex\TeX
    \let\Oldlatex\LaTeX
    \renewcommand{\TeX}{\textrm{\Oldtex}}
    \renewcommand{\LaTeX}{\textrm{\Oldlatex}}
    % Document parameters
    % Document title
    \title{Лабораторная работа №1 \\ Алгоритм обращения матрицы}
    
    
    
    
    
% Pygments definitions
\makeatletter
\def\PY@reset{\let\PY@it=\relax \let\PY@bf=\relax%
    \let\PY@ul=\relax \let\PY@tc=\relax%
    \let\PY@bc=\relax \let\PY@ff=\relax}
\def\PY@tok#1{\csname PY@tok@#1\endcsname}
\def\PY@toks#1+{\ifx\relax#1\empty\else%
    \PY@tok{#1}\expandafter\PY@toks\fi}
\def\PY@do#1{\PY@bc{\PY@tc{\PY@ul{%
    \PY@it{\PY@bf{\PY@ff{#1}}}}}}}
\def\PY#1#2{\PY@reset\PY@toks#1+\relax+\PY@do{#2}}

\expandafter\def\csname PY@tok@w\endcsname{\def\PY@tc##1{\textcolor[rgb]{0.73,0.73,0.73}{##1}}}
\expandafter\def\csname PY@tok@c\endcsname{\let\PY@it=\textit\def\PY@tc##1{\textcolor[rgb]{0.25,0.50,0.50}{##1}}}
\expandafter\def\csname PY@tok@cp\endcsname{\def\PY@tc##1{\textcolor[rgb]{0.74,0.48,0.00}{##1}}}
\expandafter\def\csname PY@tok@k\endcsname{\let\PY@bf=\textbf\def\PY@tc##1{\textcolor[rgb]{0.00,0.50,0.00}{##1}}}
\expandafter\def\csname PY@tok@kp\endcsname{\def\PY@tc##1{\textcolor[rgb]{0.00,0.50,0.00}{##1}}}
\expandafter\def\csname PY@tok@kt\endcsname{\def\PY@tc##1{\textcolor[rgb]{0.69,0.00,0.25}{##1}}}
\expandafter\def\csname PY@tok@o\endcsname{\def\PY@tc##1{\textcolor[rgb]{0.40,0.40,0.40}{##1}}}
\expandafter\def\csname PY@tok@ow\endcsname{\let\PY@bf=\textbf\def\PY@tc##1{\textcolor[rgb]{0.67,0.13,1.00}{##1}}}
\expandafter\def\csname PY@tok@nb\endcsname{\def\PY@tc##1{\textcolor[rgb]{0.00,0.50,0.00}{##1}}}
\expandafter\def\csname PY@tok@nf\endcsname{\def\PY@tc##1{\textcolor[rgb]{0.00,0.00,1.00}{##1}}}
\expandafter\def\csname PY@tok@nc\endcsname{\let\PY@bf=\textbf\def\PY@tc##1{\textcolor[rgb]{0.00,0.00,1.00}{##1}}}
\expandafter\def\csname PY@tok@nn\endcsname{\let\PY@bf=\textbf\def\PY@tc##1{\textcolor[rgb]{0.00,0.00,1.00}{##1}}}
\expandafter\def\csname PY@tok@ne\endcsname{\let\PY@bf=\textbf\def\PY@tc##1{\textcolor[rgb]{0.82,0.25,0.23}{##1}}}
\expandafter\def\csname PY@tok@nv\endcsname{\def\PY@tc##1{\textcolor[rgb]{0.10,0.09,0.49}{##1}}}
\expandafter\def\csname PY@tok@no\endcsname{\def\PY@tc##1{\textcolor[rgb]{0.53,0.00,0.00}{##1}}}
\expandafter\def\csname PY@tok@nl\endcsname{\def\PY@tc##1{\textcolor[rgb]{0.63,0.63,0.00}{##1}}}
\expandafter\def\csname PY@tok@ni\endcsname{\let\PY@bf=\textbf\def\PY@tc##1{\textcolor[rgb]{0.60,0.60,0.60}{##1}}}
\expandafter\def\csname PY@tok@na\endcsname{\def\PY@tc##1{\textcolor[rgb]{0.49,0.56,0.16}{##1}}}
\expandafter\def\csname PY@tok@nt\endcsname{\let\PY@bf=\textbf\def\PY@tc##1{\textcolor[rgb]{0.00,0.50,0.00}{##1}}}
\expandafter\def\csname PY@tok@nd\endcsname{\def\PY@tc##1{\textcolor[rgb]{0.67,0.13,1.00}{##1}}}
\expandafter\def\csname PY@tok@s\endcsname{\def\PY@tc##1{\textcolor[rgb]{0.73,0.13,0.13}{##1}}}
\expandafter\def\csname PY@tok@sd\endcsname{\let\PY@it=\textit\def\PY@tc##1{\textcolor[rgb]{0.73,0.13,0.13}{##1}}}
\expandafter\def\csname PY@tok@si\endcsname{\let\PY@bf=\textbf\def\PY@tc##1{\textcolor[rgb]{0.73,0.40,0.53}{##1}}}
\expandafter\def\csname PY@tok@se\endcsname{\let\PY@bf=\textbf\def\PY@tc##1{\textcolor[rgb]{0.73,0.40,0.13}{##1}}}
\expandafter\def\csname PY@tok@sr\endcsname{\def\PY@tc##1{\textcolor[rgb]{0.73,0.40,0.53}{##1}}}
\expandafter\def\csname PY@tok@ss\endcsname{\def\PY@tc##1{\textcolor[rgb]{0.10,0.09,0.49}{##1}}}
\expandafter\def\csname PY@tok@sx\endcsname{\def\PY@tc##1{\textcolor[rgb]{0.00,0.50,0.00}{##1}}}
\expandafter\def\csname PY@tok@m\endcsname{\def\PY@tc##1{\textcolor[rgb]{0.40,0.40,0.40}{##1}}}
\expandafter\def\csname PY@tok@gh\endcsname{\let\PY@bf=\textbf\def\PY@tc##1{\textcolor[rgb]{0.00,0.00,0.50}{##1}}}
\expandafter\def\csname PY@tok@gu\endcsname{\let\PY@bf=\textbf\def\PY@tc##1{\textcolor[rgb]{0.50,0.00,0.50}{##1}}}
\expandafter\def\csname PY@tok@gd\endcsname{\def\PY@tc##1{\textcolor[rgb]{0.63,0.00,0.00}{##1}}}
\expandafter\def\csname PY@tok@gi\endcsname{\def\PY@tc##1{\textcolor[rgb]{0.00,0.63,0.00}{##1}}}
\expandafter\def\csname PY@tok@gr\endcsname{\def\PY@tc##1{\textcolor[rgb]{1.00,0.00,0.00}{##1}}}
\expandafter\def\csname PY@tok@ge\endcsname{\let\PY@it=\textit}
\expandafter\def\csname PY@tok@gs\endcsname{\let\PY@bf=\textbf}
\expandafter\def\csname PY@tok@gp\endcsname{\let\PY@bf=\textbf\def\PY@tc##1{\textcolor[rgb]{0.00,0.00,0.50}{##1}}}
\expandafter\def\csname PY@tok@go\endcsname{\def\PY@tc##1{\textcolor[rgb]{0.53,0.53,0.53}{##1}}}
\expandafter\def\csname PY@tok@gt\endcsname{\def\PY@tc##1{\textcolor[rgb]{0.00,0.27,0.87}{##1}}}
\expandafter\def\csname PY@tok@err\endcsname{\def\PY@bc##1{\setlength{\fboxsep}{0pt}\fcolorbox[rgb]{1.00,0.00,0.00}{1,1,1}{\strut ##1}}}
\expandafter\def\csname PY@tok@kc\endcsname{\let\PY@bf=\textbf\def\PY@tc##1{\textcolor[rgb]{0.00,0.50,0.00}{##1}}}
\expandafter\def\csname PY@tok@kd\endcsname{\let\PY@bf=\textbf\def\PY@tc##1{\textcolor[rgb]{0.00,0.50,0.00}{##1}}}
\expandafter\def\csname PY@tok@kn\endcsname{\let\PY@bf=\textbf\def\PY@tc##1{\textcolor[rgb]{0.00,0.50,0.00}{##1}}}
\expandafter\def\csname PY@tok@kr\endcsname{\let\PY@bf=\textbf\def\PY@tc##1{\textcolor[rgb]{0.00,0.50,0.00}{##1}}}
\expandafter\def\csname PY@tok@bp\endcsname{\def\PY@tc##1{\textcolor[rgb]{0.00,0.50,0.00}{##1}}}
\expandafter\def\csname PY@tok@fm\endcsname{\def\PY@tc##1{\textcolor[rgb]{0.00,0.00,1.00}{##1}}}
\expandafter\def\csname PY@tok@vc\endcsname{\def\PY@tc##1{\textcolor[rgb]{0.10,0.09,0.49}{##1}}}
\expandafter\def\csname PY@tok@vg\endcsname{\def\PY@tc##1{\textcolor[rgb]{0.10,0.09,0.49}{##1}}}
\expandafter\def\csname PY@tok@vi\endcsname{\def\PY@tc##1{\textcolor[rgb]{0.10,0.09,0.49}{##1}}}
\expandafter\def\csname PY@tok@vm\endcsname{\def\PY@tc##1{\textcolor[rgb]{0.10,0.09,0.49}{##1}}}
\expandafter\def\csname PY@tok@sa\endcsname{\def\PY@tc##1{\textcolor[rgb]{0.73,0.13,0.13}{##1}}}
\expandafter\def\csname PY@tok@sb\endcsname{\def\PY@tc##1{\textcolor[rgb]{0.73,0.13,0.13}{##1}}}
\expandafter\def\csname PY@tok@sc\endcsname{\def\PY@tc##1{\textcolor[rgb]{0.73,0.13,0.13}{##1}}}
\expandafter\def\csname PY@tok@dl\endcsname{\def\PY@tc##1{\textcolor[rgb]{0.73,0.13,0.13}{##1}}}
\expandafter\def\csname PY@tok@s2\endcsname{\def\PY@tc##1{\textcolor[rgb]{0.73,0.13,0.13}{##1}}}
\expandafter\def\csname PY@tok@sh\endcsname{\def\PY@tc##1{\textcolor[rgb]{0.73,0.13,0.13}{##1}}}
\expandafter\def\csname PY@tok@s1\endcsname{\def\PY@tc##1{\textcolor[rgb]{0.73,0.13,0.13}{##1}}}
\expandafter\def\csname PY@tok@mb\endcsname{\def\PY@tc##1{\textcolor[rgb]{0.40,0.40,0.40}{##1}}}
\expandafter\def\csname PY@tok@mf\endcsname{\def\PY@tc##1{\textcolor[rgb]{0.40,0.40,0.40}{##1}}}
\expandafter\def\csname PY@tok@mh\endcsname{\def\PY@tc##1{\textcolor[rgb]{0.40,0.40,0.40}{##1}}}
\expandafter\def\csname PY@tok@mi\endcsname{\def\PY@tc##1{\textcolor[rgb]{0.40,0.40,0.40}{##1}}}
\expandafter\def\csname PY@tok@il\endcsname{\def\PY@tc##1{\textcolor[rgb]{0.40,0.40,0.40}{##1}}}
\expandafter\def\csname PY@tok@mo\endcsname{\def\PY@tc##1{\textcolor[rgb]{0.40,0.40,0.40}{##1}}}
\expandafter\def\csname PY@tok@ch\endcsname{\let\PY@it=\textit\def\PY@tc##1{\textcolor[rgb]{0.25,0.50,0.50}{##1}}}
\expandafter\def\csname PY@tok@cm\endcsname{\let\PY@it=\textit\def\PY@tc##1{\textcolor[rgb]{0.25,0.50,0.50}{##1}}}
\expandafter\def\csname PY@tok@cpf\endcsname{\let\PY@it=\textit\def\PY@tc##1{\textcolor[rgb]{0.25,0.50,0.50}{##1}}}
\expandafter\def\csname PY@tok@c1\endcsname{\let\PY@it=\textit\def\PY@tc##1{\textcolor[rgb]{0.25,0.50,0.50}{##1}}}
\expandafter\def\csname PY@tok@cs\endcsname{\let\PY@it=\textit\def\PY@tc##1{\textcolor[rgb]{0.25,0.50,0.50}{##1}}}

\def\PYZbs{\char`\\}
\def\PYZus{\char`\_}
\def\PYZob{\char`\{}
\def\PYZcb{\char`\}}
\def\PYZca{\char`\^}
\def\PYZam{\char`\&}
\def\PYZlt{\char`\<}
\def\PYZgt{\char`\>}
\def\PYZsh{\char`\#}
\def\PYZpc{\char`\%}
\def\PYZdl{\char`\$}
\def\PYZhy{\char`\-}
\def\PYZsq{\char`\'}
\def\PYZdq{\char`\"}
\def\PYZti{\char`\~}
% for compatibility with earlier versions
\def\PYZat{@}
\def\PYZlb{[}
\def\PYZrb{]}
\makeatother


    % For linebreaks inside Verbatim environment from package fancyvrb. 
    \makeatletter
        \newbox\Wrappedcontinuationbox 
        \newbox\Wrappedvisiblespacebox 
        \newcommand*\Wrappedvisiblespace {\textcolor{red}{\textvisiblespace}} 
        \newcommand*\Wrappedcontinuationsymbol {\textcolor{red}{\llap{\tiny$\m@th\hookrightarrow$}}} 
        \newcommand*\Wrappedcontinuationindent {3ex } 
        \newcommand*\Wrappedafterbreak {\kern\Wrappedcontinuationindent\copy\Wrappedcontinuationbox} 
        % Take advantage of the already applied Pygments mark-up to insert 
        % potential linebreaks for TeX processing. 
        %        {, <, #, %, $, ' and ": go to next line. 
        %        _, }, ^, &, >, - and ~: stay at end of broken line. 
        % Use of \textquotesingle for straight quote. 
        \newcommand*\Wrappedbreaksatspecials {% 
            \def\PYGZus{\discretionary{\char`\_}{\Wrappedafterbreak}{\char`\_}}% 
            \def\PYGZob{\discretionary{}{\Wrappedafterbreak\char`\{}{\char`\{}}% 
            \def\PYGZcb{\discretionary{\char`\}}{\Wrappedafterbreak}{\char`\}}}% 
            \def\PYGZca{\discretionary{\char`\^}{\Wrappedafterbreak}{\char`\^}}% 
            \def\PYGZam{\discretionary{\char`\&}{\Wrappedafterbreak}{\char`\&}}% 
            \def\PYGZlt{\discretionary{}{\Wrappedafterbreak\char`\<}{\char`\<}}% 
            \def\PYGZgt{\discretionary{\char`\>}{\Wrappedafterbreak}{\char`\>}}% 
            \def\PYGZsh{\discretionary{}{\Wrappedafterbreak\char`\#}{\char`\#}}% 
            \def\PYGZpc{\discretionary{}{\Wrappedafterbreak\char`\%}{\char`\%}}% 
            \def\PYGZdl{\discretionary{}{\Wrappedafterbreak\char`\$}{\char`\$}}% 
            \def\PYGZhy{\discretionary{\char`\-}{\Wrappedafterbreak}{\char`\-}}% 
            \def\PYGZsq{\discretionary{}{\Wrappedafterbreak\textquotesingle}{\textquotesingle}}% 
            \def\PYGZdq{\discretionary{}{\Wrappedafterbreak\char`\"}{\char`\"}}% 
            \def\PYGZti{\discretionary{\char`\~}{\Wrappedafterbreak}{\char`\~}}% 
        } 
        % Some characters . , ; ? ! / are not pygmentized. 
        % This macro makes them "active" and they will insert potential linebreaks 
        \newcommand*\Wrappedbreaksatpunct {% 
            \lccode`\~`\.\lowercase{\def~}{\discretionary{\hbox{\char`\.}}{\Wrappedafterbreak}{\hbox{\char`\.}}}% 
            \lccode`\~`\,\lowercase{\def~}{\discretionary{\hbox{\char`\,}}{\Wrappedafterbreak}{\hbox{\char`\,}}}% 
            \lccode`\~`\;\lowercase{\def~}{\discretionary{\hbox{\char`\;}}{\Wrappedafterbreak}{\hbox{\char`\;}}}% 
            \lccode`\~`\:\lowercase{\def~}{\discretionary{\hbox{\char`\:}}{\Wrappedafterbreak}{\hbox{\char`\:}}}% 
            \lccode`\~`\?\lowercase{\def~}{\discretionary{\hbox{\char`\?}}{\Wrappedafterbreak}{\hbox{\char`\?}}}% 
            \lccode`\~`\!\lowercase{\def~}{\discretionary{\hbox{\char`\!}}{\Wrappedafterbreak}{\hbox{\char`\!}}}% 
            \lccode`\~`\/\lowercase{\def~}{\discretionary{\hbox{\char`\/}}{\Wrappedafterbreak}{\hbox{\char`\/}}}% 
            \catcode`\.\active
            \catcode`\,\active 
            \catcode`\;\active
            \catcode`\:\active
            \catcode`\?\active
            \catcode`\!\active
            \catcode`\/\active 
            \lccode`\~`\~ 	
        }
    \makeatother

    \let\OriginalVerbatim=\Verbatim
    \makeatletter
    \renewcommand{\Verbatim}[1][1]{%
        %\parskip\z@skip
        \sbox\Wrappedcontinuationbox {\Wrappedcontinuationsymbol}%
        \sbox\Wrappedvisiblespacebox {\FV@SetupFont\Wrappedvisiblespace}%
        \def\FancyVerbFormatLine ##1{\hsize\linewidth
            \vtop{\raggedright\hyphenpenalty\z@\exhyphenpenalty\z@
                \doublehyphendemerits\z@\finalhyphendemerits\z@
                \strut ##1\strut}%
        }%
        % If the linebreak is at a space, the latter will be displayed as visible
        % space at end of first line, and a continuation symbol starts next line.
        % Stretch/shrink are however usually zero for typewriter font.
        \def\FV@Space {%
            \nobreak\hskip\z@ plus\fontdimen3\font minus\fontdimen4\font
            \discretionary{\copy\Wrappedvisiblespacebox}{\Wrappedafterbreak}
            {\kern\fontdimen2\font}%
        }%
        
        % Allow breaks at special characters using \PYG... macros.
        \Wrappedbreaksatspecials
        % Breaks at punctuation characters . , ; ? ! and / need catcode=\active 	
        \OriginalVerbatim[#1,codes*=\Wrappedbreaksatpunct]%
    }
    \makeatother

    % Exact colors from NB
    \definecolor{incolor}{HTML}{303F9F}
    \definecolor{outcolor}{HTML}{D84315}
    \definecolor{cellborder}{HTML}{CFCFCF}
    \definecolor{cellbackground}{HTML}{F7F7F7}
    
    % prompt
    \makeatletter
    \newcommand{\boxspacing}{\kern\kvtcb@left@rule\kern\kvtcb@boxsep}
    \makeatother
    \newcommand{\prompt}[4]{
        {\ttfamily\llap{{\color{#2}[#3]:\hspace{3pt}#4}}\vspace{-\baselineskip}}
    }
    

    
    % Prevent overflowing lines due to hard-to-break entities
    \sloppy 
    % Setup hyperref package
    \hypersetup{
      breaklinks=true,  % so long urls are correctly broken across lines
      colorlinks=true,
      urlcolor=urlcolor,
      linkcolor=linkcolor,
      citecolor=citecolor,
      }
    % Slightly bigger margins than the latex defaults
    
    \geometry{verbose,tmargin=1in,bmargin=1in,lmargin=1in,rmargin=1in}
    
    

\begin{document}
    
    \maketitle
    


Выполнил: Совпель Дмитрий, гр. 853501

    \hypertarget{ux440ux435ux430ux43bux438ux437ux430ux446ux438ux44f-ux43cux435ux442ux43eux434ux430}{%
\section{Реализация
метода}\label{ux440ux435ux430ux43bux438ux437ux430ux446ux438ux44f-ux43cux435ux442ux43eux434ux430}}

    Импортируем необходимые библиотеки.

    \begin{tcolorbox}[breakable, size=fbox, boxrule=1pt, pad at break*=1mm,colback=cellbackground, colframe=cellborder]
\prompt{In}{incolor}{8}{\boxspacing}
\begin{Verbatim}[commandchars=\\\{\}]
\PY{k+kn}{import} \PY{n+nn}{warnings}
\PY{k+kn}{import} \PY{n+nn}{typing} \PY{k}{as} \PY{n+nn}{tp}

\PY{k+kn}{import} \PY{n+nn}{numpy} \PY{k}{as} \PY{n+nn}{np}
\PY{k+kn}{import} \PY{n+nn}{scipy}\PY{n+nn}{.}\PY{n+nn}{linalg} \PY{k}{as} \PY{n+nn}{sla}
\end{Verbatim}
\end{tcolorbox}

    Реализуем функции, необходимые для построения алгоритмы
\texttt{compute\_alpha} и \texttt{compute\_inverse\_step}. Затем
реализуем сам алгоритм обращения матрицы \texttt{inverse}.

    \begin{tcolorbox}[breakable, size=fbox, boxrule=1pt, pad at break*=1mm,colback=cellbackground, colframe=cellborder]
\prompt{In}{incolor}{72}{\boxspacing}
\begin{Verbatim}[commandchars=\\\{\}]
\PY{k}{def} \PY{n+nf}{compute\PYZus{}alpha}\PY{p}{(}\PY{n}{inverse\PYZus{}matrix}\PY{p}{:} \PY{n}{np}\PY{o}{.}\PY{n}{array}\PY{p}{,} \PY{n}{column}\PY{p}{:} \PY{n}{np}\PY{o}{.}\PY{n}{array}\PY{p}{,} \PY{n}{column\PYZus{}no}\PY{p}{:} \PY{n+nb}{int}\PY{p}{)} \PY{o}{\PYZhy{}}\PY{o}{\PYZgt{}} \PY{n+nb}{float}\PY{p}{:}
    \PY{k}{return} \PY{n}{inverse\PYZus{}matrix}\PY{p}{[}\PY{n}{column\PYZus{}no}\PY{p}{]} \PY{o}{@} \PY{n}{column}

\PY{k}{def} \PY{n+nf}{compute\PYZus{}inverse\PYZus{}step}\PY{p}{(}\PY{n}{previous\PYZus{}inverse\PYZus{}matrix}\PY{p}{:} \PY{n}{np}\PY{o}{.}\PY{n}{array}\PY{p}{,} \PY{n}{column}\PY{p}{:} \PY{n}{np}\PY{o}{.}\PY{n}{array}\PY{p}{,} \PY{n}{column\PYZus{}no}\PY{p}{:} \PY{n+nb}{int}\PY{p}{)} \PY{o}{\PYZhy{}}\PY{o}{\PYZgt{}} \PY{n}{np}\PY{o}{.}\PY{n}{array}\PY{p}{:}
    \PY{n}{d} \PY{o}{=} \PY{n}{previous\PYZus{}inverse\PYZus{}matrix} \PY{o}{@} \PY{n}{column}
    \PY{n}{d\PYZus{}k} \PY{o}{=} \PY{n}{d}\PY{p}{[}\PY{n}{column\PYZus{}no}\PY{p}{]}
    \PY{n}{d}\PY{p}{[}\PY{n}{column\PYZus{}no}\PY{p}{]} \PY{o}{=} \PY{o}{\PYZhy{}}\PY{l+m+mi}{1}
    \PY{n}{d} \PY{o}{/}\PY{o}{=} \PY{o}{\PYZhy{}}\PY{n}{d\PYZus{}k}
    
    \PY{n}{D} \PY{o}{=} \PY{n}{np}\PY{o}{.}\PY{n}{eye}\PY{p}{(}\PY{n}{previous\PYZus{}inverse\PYZus{}matrix}\PY{o}{.}\PY{n}{shape}\PY{p}{[}\PY{l+m+mi}{0}\PY{p}{]}\PY{p}{)}
    \PY{n}{D}\PY{p}{[}\PY{p}{:}\PY{p}{,} \PY{n}{column\PYZus{}no}\PY{p}{]} \PY{o}{=} \PY{n}{d}
    
    \PY{k}{return} \PY{n}{D} \PY{o}{@} \PY{n}{previous\PYZus{}inverse\PYZus{}matrix}

\PY{k}{def} \PY{n+nf}{inverse}\PY{p}{(}\PY{n}{matrix}\PY{p}{:} \PY{n}{np}\PY{o}{.}\PY{n}{array}\PY{p}{,} \PY{n}{verbose}\PY{p}{:} \PY{n+nb}{bool} \PY{o}{=} \PY{k+kc}{False}\PY{p}{)} \PY{o}{\PYZhy{}}\PY{o}{\PYZgt{}} \PY{n}{np}\PY{o}{.}\PY{n}{array}\PY{p}{:}
    \PY{n}{current\PYZus{}inverse\PYZus{}matrix} \PY{o}{=} \PY{n}{np}\PY{o}{.}\PY{n}{eye}\PY{p}{(}\PY{n}{matrix}\PY{o}{.}\PY{n}{shape}\PY{p}{[}\PY{l+m+mi}{0}\PY{p}{]}\PY{p}{)}
    \PY{n}{iteration} \PY{o}{=} \PY{l+m+mi}{0}
    
    \PY{n}{indexes\PYZus{}set} \PY{o}{=} \PY{n+nb}{set}\PY{p}{(}\PY{n+nb}{range}\PY{p}{(}\PY{n}{matrix}\PY{o}{.}\PY{n}{shape}\PY{p}{[}\PY{l+m+mi}{0}\PY{p}{]}\PY{p}{)}\PY{p}{)}
    \PY{n}{indexes\PYZus{}order} \PY{o}{=} \PY{p}{[}\PY{l+m+mi}{0}\PY{p}{]} \PY{o}{*} \PY{n}{matrix}\PY{o}{.}\PY{n}{shape}\PY{p}{[}\PY{l+m+mi}{0}\PY{p}{]}
    \PY{k}{if} \PY{n}{verbose}\PY{p}{:}
        \PY{n+nb}{print}\PY{p}{(}\PY{l+s+sa}{f}\PY{l+s+s1}{\PYZsq{}}\PY{l+s+s1}{Initial unused indexes set }\PY{l+s+s1}{\PYZob{}}\PY{l+s+s1}{set([element + 1 for element in indexes\PYZus{}set])\PYZcb{}}\PY{l+s+s1}{\PYZsq{}}\PY{p}{)}
        \PY{n+nb}{print}\PY{p}{(}\PY{l+s+sa}{f}\PY{l+s+s1}{\PYZsq{}}\PY{l+s+s1}{Initial matrix }\PY{l+s+se}{\PYZbs{}n}\PY{l+s+s1}{ }\PY{l+s+si}{\PYZob{}current\PYZus{}inverse\PYZus{}matrix\PYZcb{}}\PY{l+s+s1}{\PYZsq{}}\PY{p}{)}
    
    \PY{k}{while} \PY{n}{iteration} \PY{o}{\PYZlt{}} \PY{n}{matrix}\PY{o}{.}\PY{n}{shape}\PY{p}{[}\PY{l+m+mi}{0}\PY{p}{]}\PY{p}{:}
        \PY{n}{is\PYZus{}singular} \PY{o}{=} \PY{k+kc}{True}
        \PY{k}{for} \PY{n}{current\PYZus{}column\PYZus{}no} \PY{o+ow}{in} \PY{n}{indexes\PYZus{}set}\PY{p}{:}
            \PY{n}{current\PYZus{}column} \PY{o}{=} \PY{n}{matrix}\PY{p}{[}\PY{p}{:} \PY{p}{,}\PY{n}{current\PYZus{}column\PYZus{}no}\PY{p}{]}
            \PY{n}{alpha} \PY{o}{=} \PY{n}{compute\PYZus{}alpha}\PY{p}{(}\PY{n}{current\PYZus{}inverse\PYZus{}matrix}\PY{p}{,} \PY{n}{current\PYZus{}column}\PY{p}{,} \PY{n}{iteration}\PY{p}{)}
            \PY{k}{if} \PY{o+ow}{not} \PY{n}{np}\PY{o}{.}\PY{n}{isclose}\PY{p}{(}\PY{n}{alpha}\PY{p}{,} \PY{l+m+mf}{0.0}\PY{p}{,} \PY{n}{rtol}\PY{o}{=}\PY{l+m+mf}{1e\PYZhy{}11}\PY{p}{)}\PY{p}{:}
                \PY{n}{is\PYZus{}singular} \PY{o}{=} \PY{k+kc}{False}
                \PY{n}{indexes\PYZus{}set}\PY{o}{.}\PY{n}{remove}\PY{p}{(}\PY{n}{current\PYZus{}column\PYZus{}no}\PY{p}{)}
                \PY{n}{indexes\PYZus{}order}\PY{p}{[}\PY{n}{current\PYZus{}column\PYZus{}no}\PY{p}{]} \PY{o}{=} \PY{n}{iteration}
                \PY{k}{break}
        \PY{k}{if} \PY{n}{is\PYZus{}singular}\PY{p}{:}
            \PY{n}{warnings}\PY{o}{.}\PY{n}{warn}\PY{p}{(}\PY{l+s+s2}{\PYZdq{}}\PY{l+s+s2}{Singular matrix}\PY{l+s+s2}{\PYZdq{}}\PY{p}{,} \PY{n}{sla}\PY{o}{.}\PY{n}{LinAlgWarning}\PY{p}{)}
            \PY{k}{return} \PY{k+kc}{None}   
        \PY{n}{current\PYZus{}inverse\PYZus{}matrix} \PY{o}{=} \PY{n}{compute\PYZus{}inverse\PYZus{}step}\PY{p}{(}\PY{n}{current\PYZus{}inverse\PYZus{}matrix}\PY{p}{,} \PY{n}{current\PYZus{}column}\PY{p}{,} \PY{n}{iteration}\PY{p}{)}
        \PY{n}{iteration} \PY{o}{+}\PY{o}{=} \PY{l+m+mi}{1}
        
        \PY{k}{if} \PY{n}{verbose}\PY{p}{:}
            \PY{n+nb}{print}\PY{p}{(}\PY{l+s+sa}{f}\PY{l+s+s1}{\PYZsq{}}\PY{l+s+s1}{Current iteration no. }\PY{l+s+si}{\PYZob{}iteration\PYZcb{}}\PY{l+s+s1}{\PYZsq{}}\PY{p}{)}
            \PY{n+nb}{print}\PY{p}{(}\PY{l+s+sa}{f}\PY{l+s+s1}{\PYZsq{}}\PY{l+s+s1}{Current indexes set }\PY{l+s+s1}{\PYZob{}}\PY{l+s+s1}{set([element + 1 for element in indexes\PYZus{}set]) if indexes\PYZus{}set else }\PY{l+s+s1}{\PYZdq{}}\PY{l+s+si}{\PYZob{}\PYZcb{}}\PY{l+s+s1}{\PYZdq{}}\PY{l+s+s1}{\PYZcb{}}\PY{l+s+s1}{\PYZsq{}}\PY{p}{)}
            \PY{n+nb}{print}\PY{p}{(}\PY{l+s+sa}{f}\PY{l+s+s1}{\PYZsq{}}\PY{l+s+s1}{Current inverse matrix }\PY{l+s+se}{\PYZbs{}n}\PY{l+s+s1}{ }\PY{l+s+si}{\PYZob{}current\PYZus{}inverse\PYZus{}matrix\PYZcb{}}\PY{l+s+s1}{\PYZsq{}}\PY{p}{)}
    \PY{k}{return} \PY{n}{current\PYZus{}inverse\PYZus{}matrix}\PY{p}{[}\PY{n}{np}\PY{o}{.}\PY{n}{array}\PY{p}{(}\PY{n}{indexes\PYZus{}order}\PY{p}{)}\PY{p}{,} \PY{p}{:}\PY{p}{]}
\end{Verbatim}
\end{tcolorbox}

    Для тестирования используем вспомогательную функцию \texttt{test}.
Функция проверяет условие \(|AA^{-1} - E|_F < \varepsilon\) (при этом по
умолчанию \(\varepsilon = 0.00001\)), и если оно не соблюдается выдает
ошибку.

    \begin{tcolorbox}[breakable, size=fbox, boxrule=1pt, pad at break*=1mm,colback=cellbackground, colframe=cellborder]
\prompt{In}{incolor}{33}{\boxspacing}
\begin{Verbatim}[commandchars=\\\{\}]
\PY{k}{def} \PY{n+nf}{test}\PY{p}{(}\PY{n}{matrix}\PY{p}{:} \PY{n}{np}\PY{o}{.}\PY{n}{array}\PY{p}{,} \PY{n}{inverse\PYZus{}matrix}\PY{p}{:} \PY{n}{np}\PY{o}{.}\PY{n}{array}\PY{p}{,} \PY{n}{eps} \PY{o}{=} \PY{l+m+mf}{1e\PYZhy{}5}\PY{p}{)} \PY{o}{\PYZhy{}}\PY{o}{\PYZgt{}} \PY{k+kc}{None}\PY{p}{:}
    \PY{k}{if} \PY{n}{inverse\PYZus{}matrix} \PY{o+ow}{is} \PY{o+ow}{not} \PY{k+kc}{None}\PY{p}{:}
        \PY{k}{assert} \PY{n}{sla}\PY{o}{.}\PY{n}{norm}\PY{p}{(}\PY{n}{matrix} \PY{o}{@} \PY{n}{inverse\PYZus{}matrix} \PY{o}{\PYZhy{}} \PY{n}{np}\PY{o}{.}\PY{n}{eye}\PY{p}{(}\PY{n}{matrix}\PY{o}{.}\PY{n}{shape}\PY{p}{[}\PY{l+m+mi}{0}\PY{p}{]}\PY{p}{)}\PY{p}{)} \PY{o}{\PYZlt{}} \PY{n}{eps}
\end{Verbatim}
\end{tcolorbox}

    \hypertarget{ux442ux435ux441ux442ux438ux440ux43eux432ux430ux43dux438ux435-ux430ux43bux433ux43eux440ux438ux442ux43cux430-ux43dux430-ux43cux430ux442ux440ux438ux446ux435-a}{%
\section{\texorpdfstring{Тестирование алгоритма на матрице
\(A\)}{Тестирование алгоритма на матрице A}}\label{ux442ux435ux441ux442ux438ux440ux43eux432ux430ux43dux438ux435-ux430ux43bux433ux43eux440ux438ux442ux43cux430-ux43dux430-ux43cux430ux442ux440ux438ux446ux435-a}}

    Матрица \(A\) имеет вид: \[A = \begin{pmatrix}
0 & 0 & 0 & 1 \\
0 & 0 & 1 & 0 \\
0 & 1 & 0 & 0 \\
1 & 0 & 0 & 0 \\
\end{pmatrix}\]

    Как видно, алгоритм корректно отрабатывает данный тестовый случай.

    \begin{tcolorbox}[breakable, size=fbox, boxrule=1pt, pad at break*=1mm,colback=cellbackground, colframe=cellborder]
\prompt{In}{incolor}{11}{\boxspacing}
\begin{Verbatim}[commandchars=\\\{\}]
\PY{n}{matrix} \PY{o}{=} \PY{n}{np}\PY{o}{.}\PY{n}{array}\PY{p}{(}\PY{p}{[}\PY{p}{[}\PY{l+m+mi}{0}\PY{p}{,} \PY{l+m+mi}{0}\PY{p}{,} \PY{l+m+mi}{0}\PY{p}{,} \PY{l+m+mi}{1}\PY{p}{]}\PY{p}{,} \PY{p}{[}\PY{l+m+mi}{0}\PY{p}{,} \PY{l+m+mi}{0}\PY{p}{,} \PY{l+m+mi}{1}\PY{p}{,} \PY{l+m+mi}{0}\PY{p}{]}\PY{p}{,} \PY{p}{[}\PY{l+m+mi}{0}\PY{p}{,} \PY{l+m+mi}{1}\PY{p}{,} \PY{l+m+mi}{0}\PY{p}{,} \PY{l+m+mi}{0}\PY{p}{]}\PY{p}{,} \PY{p}{[}\PY{l+m+mi}{1}\PY{p}{,} \PY{l+m+mi}{0}\PY{p}{,} \PY{l+m+mi}{0}\PY{p}{,} \PY{l+m+mi}{0}\PY{p}{]}\PY{p}{]}\PY{p}{)}
\PY{n}{inverse\PYZus{}matrix} \PY{o}{=} \PY{n}{inverse}\PY{p}{(}\PY{n}{matrix}\PY{p}{)}
\PY{n}{test}\PY{p}{(}\PY{n}{matrix}\PY{p}{,} \PY{n}{inverse\PYZus{}matrix}\PY{p}{)}
\end{Verbatim}
\end{tcolorbox}

    \hypertarget{ux442ux435ux441ux442ux438ux440ux43eux432ux430ux43dux438ux435-ux430ux43bux433ux43eux440ux438ux442ux43cux430-ux43dux430-ux43cux430ux442ux440ux438ux446ux435-b_varepsilon-ux434ux43bux44f-ux438ux441ux43bux435ux434ux43eux432ux430ux43dux438ux44f-ux442ux43eux447ux43dux43eux441ux442ux438-ux43cux435ux442ux43eux434ux430}{%
\section{\texorpdfstring{Тестирование алгоритма на матрице
\(B_{\varepsilon}\) для иследования точности
метода}{Тестирование алгоритма на матрице B\_\{\textbackslash{}varepsilon\} для иследования точности метода}}\label{ux442ux435ux441ux442ux438ux440ux43eux432ux430ux43dux438ux435-ux430ux43bux433ux43eux440ux438ux442ux43cux430-ux43dux430-ux43cux430ux442ux440ux438ux446ux435-b_varepsilon-ux434ux43bux44f-ux438ux441ux43bux435ux434ux43eux432ux430ux43dux438ux44f-ux442ux43eux447ux43dux43eux441ux442ux438-ux43cux435ux442ux43eux434ux430}}

    Рассмотрим матрицы вида \(B_{eps}\) \[B_{eps} = \begin{pmatrix}
2 & 8 & -1 & 4 & 5 & 6 \\ 
1 & -9 & 2 & -3 & 1 & -2 \\                          
3 & -1 & 1 & 1 + eps & 6 & 4 \\
0 & 1 & 0 & 1 & 0 & 2 \\
1 & 2 & -1 & 4 & 2 & 3 \\
-3 & 2 & 1 & 0 & 0 & 0 
\end{pmatrix}\] для различных
\(eps = \{1, 0.1, 0.0000001, 0.000000001, 0.000000000000001\}\).

    \begin{tcolorbox}[breakable, size=fbox, boxrule=1pt, pad at break*=1mm,colback=cellbackground, colframe=cellborder]
\prompt{In}{incolor}{15}{\boxspacing}
\begin{Verbatim}[commandchars=\\\{\}]
\PY{n}{B} \PY{o}{=} \PY{k}{lambda} \PY{n}{eps}\PY{p}{:} \PY{n}{np}\PY{o}{.}\PY{n}{array}\PY{p}{(}\PY{p}{[}\PY{p}{[}\PY{l+m+mi}{2}\PY{p}{,} \PY{l+m+mi}{8}\PY{p}{,} \PY{o}{\PYZhy{}}\PY{l+m+mi}{1}\PY{p}{,} \PY{l+m+mi}{4}\PY{p}{,} \PY{l+m+mi}{5}\PY{p}{,} \PY{l+m+mi}{6}\PY{p}{]}\PY{p}{,} 
                          \PY{p}{[}\PY{l+m+mi}{1}\PY{p}{,} \PY{o}{\PYZhy{}}\PY{l+m+mi}{9}\PY{p}{,} \PY{l+m+mi}{2}\PY{p}{,} \PY{o}{\PYZhy{}}\PY{l+m+mi}{3}\PY{p}{,} \PY{l+m+mi}{1}\PY{p}{,} \PY{o}{\PYZhy{}}\PY{l+m+mi}{2}\PY{p}{]}\PY{p}{,} 
                          \PY{p}{[}\PY{l+m+mi}{3}\PY{p}{,} \PY{o}{\PYZhy{}}\PY{l+m+mi}{1}\PY{p}{,} \PY{l+m+mi}{1}\PY{p}{,} \PY{l+m+mi}{1} \PY{o}{+} \PY{n}{eps}\PY{p}{,} \PY{l+m+mi}{6}\PY{p}{,} \PY{l+m+mi}{4}\PY{p}{]}\PY{p}{,}
                          \PY{p}{[}\PY{l+m+mi}{0}\PY{p}{,} \PY{l+m+mi}{1}\PY{p}{,} \PY{l+m+mi}{0}\PY{p}{,} \PY{l+m+mi}{1}\PY{p}{,} \PY{l+m+mi}{0}\PY{p}{,} \PY{l+m+mi}{2}\PY{p}{]}\PY{p}{,} 
                          \PY{p}{[}\PY{l+m+mi}{1}\PY{p}{,} \PY{l+m+mi}{2}\PY{p}{,} \PY{o}{\PYZhy{}}\PY{l+m+mi}{1}\PY{p}{,} \PY{l+m+mi}{4}\PY{p}{,} \PY{l+m+mi}{2}\PY{p}{,} \PY{l+m+mi}{3}\PY{p}{]}\PY{p}{,}
                          \PY{p}{[}\PY{o}{\PYZhy{}}\PY{l+m+mi}{3}\PY{p}{,} \PY{l+m+mi}{2}\PY{p}{,} \PY{l+m+mi}{1}\PY{p}{,} \PY{l+m+mi}{0}\PY{p}{,} \PY{l+m+mi}{0}\PY{p}{,} \PY{l+m+mi}{0}\PY{p}{]}\PY{p}{]}\PY{p}{)}
\PY{n}{epses} \PY{o}{=} \PY{p}{[}\PY{l+m+mi}{1}\PY{p}{,} \PY{l+m+mf}{0.1}\PY{p}{,} \PY{l+m+mf}{1e\PYZhy{}6}\PY{p}{,} \PY{l+m+mf}{1e\PYZhy{}9}\PY{p}{,} \PY{l+m+mf}{1e\PYZhy{}15}\PY{p}{]}
\end{Verbatim}
\end{tcolorbox}

    \hypertarget{eps-1}{%
\subsection{eps = 1}\label{eps-1}}

Рассмотрим \(B_{eps}\) при \(eps = 1\). Как видно, данную ситуацию
алгоритм отрабатывает корректно.

    \begin{tcolorbox}[breakable, size=fbox, boxrule=1pt, pad at break*=1mm,colback=cellbackground, colframe=cellborder]
\prompt{In}{incolor}{36}{\boxspacing}
\begin{Verbatim}[commandchars=\\\{\}]
\PY{n}{matrix} \PY{o}{=} \PY{n}{B}\PY{p}{(}\PY{n}{epses}\PY{p}{[}\PY{l+m+mi}{0}\PY{p}{]}\PY{p}{)}
\PY{n}{inverse\PYZus{}matrix} \PY{o}{=} \PY{n}{inverse}\PY{p}{(}\PY{n}{matrix}\PY{p}{)}
\PY{n}{test}\PY{p}{(}\PY{n}{matrix}\PY{p}{,} \PY{n}{inverse\PYZus{}matrix}\PY{p}{)}
\end{Verbatim}
\end{tcolorbox}

    \hypertarget{eps-0.1}{%
\subsection{eps = 0.1}\label{eps-0.1}}

Рассмотрим \(B_{eps}\) при \(eps = 0.1\). Как видно, данную ситуацию
алгоритм также отрабатывает корректно.

    \begin{tcolorbox}[breakable, size=fbox, boxrule=1pt, pad at break*=1mm,colback=cellbackground, colframe=cellborder]
\prompt{In}{incolor}{22}{\boxspacing}
\begin{Verbatim}[commandchars=\\\{\}]
\PY{n}{matrix} \PY{o}{=} \PY{n}{B}\PY{p}{(}\PY{n}{epses}\PY{p}{[}\PY{l+m+mi}{1}\PY{p}{]}\PY{p}{)}
\PY{n}{inverse\PYZus{}matrix} \PY{o}{=} \PY{n}{inverse}\PY{p}{(}\PY{n}{matrix}\PY{p}{)}
\PY{n}{test}\PY{p}{(}\PY{n}{matrix}\PY{p}{,} \PY{n}{inverse\PYZus{}matrix}\PY{p}{)}
\end{Verbatim}
\end{tcolorbox}

    \hypertarget{eps-0.000001}{%
\subsection{eps = 0.000001}\label{eps-0.000001}}

Рассмотрим \(B_{eps}\) при \(eps = 0.000001\). В данном случае, условие
\(|BB^{-1} - E|_F < 0.00001\) не соблюдается. Попробуем условие
\(|BB^{-1} - E|_F < 0.1\).

    \begin{tcolorbox}[breakable, size=fbox, boxrule=1pt, pad at break*=1mm,colback=cellbackground, colframe=cellborder]
\prompt{In}{incolor}{35}{\boxspacing}
\begin{Verbatim}[commandchars=\\\{\}]
\PY{n}{matrix} \PY{o}{=} \PY{n}{B}\PY{p}{(}\PY{n}{epses}\PY{p}{[}\PY{l+m+mi}{2}\PY{p}{]}\PY{p}{)}
\PY{n}{inverse\PYZus{}matrix} \PY{o}{=} \PY{n}{inverse}\PY{p}{(}\PY{n}{matrix}\PY{p}{)}
\PY{n}{test}\PY{p}{(}\PY{n}{matrix}\PY{p}{,} \PY{n}{inverse\PYZus{}matrix}\PY{p}{)}
\end{Verbatim}
\end{tcolorbox}

    \begin{Verbatim}[commandchars=\\\{\}, frame=single, framerule=2mm, rulecolor=\color{outerrorbackground}]
\textcolor{ansi-red}{---------------------------------------------------------------------------}
\textcolor{ansi-red}{AssertionError}                            Traceback (most recent call last)
\textcolor{ansi-green}{<ipython-input-35-af6f3b8d0591>} in \textcolor{ansi-cyan}{<module>}
\textcolor{ansi-green-intense}{\textbf{      1}} matrix \textcolor{ansi-blue}{=} B\textcolor{ansi-blue}{(}epses\textcolor{ansi-blue}{[}\textcolor{ansi-cyan}{2}\textcolor{ansi-blue}{]}\textcolor{ansi-blue}{)}
\textcolor{ansi-green-intense}{\textbf{      2}} inverse\_matrix \textcolor{ansi-blue}{=} inverse\textcolor{ansi-blue}{(}matrix\textcolor{ansi-blue}{)}
\textcolor{ansi-green}{----> 3}\textcolor{ansi-red}{ }test\textcolor{ansi-blue}{(}matrix\textcolor{ansi-blue}{,} inverse\_matrix\textcolor{ansi-blue}{)}

\textcolor{ansi-green}{<ipython-input-33-327f4077c160>} in \textcolor{ansi-cyan}{test}\textcolor{ansi-blue}{(matrix, inverse\_matrix, eps)}
\textcolor{ansi-green-intense}{\textbf{      1}} \textcolor{ansi-green}{def} test\textcolor{ansi-blue}{(}matrix\textcolor{ansi-blue}{:} np\textcolor{ansi-blue}{.}array\textcolor{ansi-blue}{,} inverse\_matrix\textcolor{ansi-blue}{:} np\textcolor{ansi-blue}{.}array\textcolor{ansi-blue}{,} eps \textcolor{ansi-blue}{=} \textcolor{ansi-cyan}{1e-5}\textcolor{ansi-blue}{)} \textcolor{ansi-blue}{->} \textcolor{ansi-green}{None}\textcolor{ansi-blue}{:}
\textcolor{ansi-green-intense}{\textbf{      2}}     \textcolor{ansi-green}{if} inverse\_matrix \textcolor{ansi-green}{is} \textcolor{ansi-green}{not} \textcolor{ansi-green}{None}\textcolor{ansi-blue}{:}
\textcolor{ansi-green}{----> 3}\textcolor{ansi-red}{         }\textcolor{ansi-green}{assert} sla\textcolor{ansi-blue}{.}norm\textcolor{ansi-blue}{(}matrix \textcolor{ansi-blue}{@} inverse\_matrix \textcolor{ansi-blue}{-} np\textcolor{ansi-blue}{.}eye\textcolor{ansi-blue}{(}matrix\textcolor{ansi-blue}{.}shape\textcolor{ansi-blue}{[}\textcolor{ansi-cyan}{0}\textcolor{ansi-blue}{]}\textcolor{ansi-blue}{)}\textcolor{ansi-blue}{)} \textcolor{ansi-blue}{<} eps

\textcolor{ansi-red}{AssertionError}: 
    \end{Verbatim}

    Как видно, условие \(|BB^{-1} - E|_F < 0.1\) соблюдается. Для сравнения
посмотрим, как справляется с этим тестовым случаем встроенная функция
библиотеки \texttt{scipy}.

    \begin{tcolorbox}[breakable, size=fbox, boxrule=1pt, pad at break*=1mm,colback=cellbackground, colframe=cellborder]
\prompt{In}{incolor}{37}{\boxspacing}
\begin{Verbatim}[commandchars=\\\{\}]
\PY{n}{matrix} \PY{o}{=} \PY{n}{B}\PY{p}{(}\PY{n}{epses}\PY{p}{[}\PY{l+m+mi}{2}\PY{p}{]}\PY{p}{)}
\PY{n}{inverse\PYZus{}matrix} \PY{o}{=} \PY{n}{inverse}\PY{p}{(}\PY{n}{matrix}\PY{p}{)}
\PY{n}{test}\PY{p}{(}\PY{n}{matrix}\PY{p}{,} \PY{n}{inverse\PYZus{}matrix}\PY{p}{,} \PY{l+m+mf}{0.1}\PY{p}{)}
\end{Verbatim}
\end{tcolorbox}

    Как видно, встроенная функция \texttt{inv} справляется с данным тестовым
случаем (то есть точность метода, заложенного в функцию \texttt{inv}
выше, чем метода, реализованного рассматриваемым алгоритмом).

    \begin{tcolorbox}[breakable, size=fbox, boxrule=1pt, pad at break*=1mm,colback=cellbackground, colframe=cellborder]
\prompt{In}{incolor}{42}{\boxspacing}
\begin{Verbatim}[commandchars=\\\{\}]
\PY{n}{matrix} \PY{o}{=} \PY{n}{B}\PY{p}{(}\PY{n}{epses}\PY{p}{[}\PY{l+m+mi}{2}\PY{p}{]}\PY{p}{)}
\PY{n}{inverse\PYZus{}matrix} \PY{o}{=} \PY{n}{sla}\PY{o}{.}\PY{n}{inv}\PY{p}{(}\PY{n}{matrix}\PY{p}{)}
\PY{n}{test}\PY{p}{(}\PY{n}{matrix}\PY{p}{,} \PY{n}{inverse\PYZus{}matrix}\PY{p}{)}
\end{Verbatim}
\end{tcolorbox}

    \hypertarget{eps-0.000000001}{%
\subsection{eps = 0.000000001}\label{eps-0.000000001}}

Рассмотрим \(B_{eps}\) при \(eps = 0.000000001\). В данном случае,
алгоритм выдает сообщение, о том что матрица является вырожденной.

    \begin{tcolorbox}[breakable, size=fbox, boxrule=1pt, pad at break*=1mm,colback=cellbackground, colframe=cellborder]
\prompt{In}{incolor}{47}{\boxspacing}
\begin{Verbatim}[commandchars=\\\{\}]
\PY{n}{matrix} \PY{o}{=} \PY{n}{B}\PY{p}{(}\PY{n}{epses}\PY{p}{[}\PY{l+m+mi}{3}\PY{p}{]}\PY{p}{)}
\PY{n}{inverse\PYZus{}matrix} \PY{o}{=} \PY{n}{inverse}\PY{p}{(}\PY{n}{matrix}\PY{p}{)}
\PY{n}{test}\PY{p}{(}\PY{n}{matrix}\PY{p}{,} \PY{n}{inverse\PYZus{}matrix}\PY{p}{)}
\end{Verbatim}
\end{tcolorbox}

    \begin{Verbatim}[commandchars=\\\{\}]
<ipython-input-9-089af5996201>:33: LinAlgWarning: Singular matrix
  warnings.warn("Singular matrix", sla.LinAlgWarning)
    \end{Verbatim}

    При этом, встроенный метод \texttt{inv} выдает корректный результат.

    \begin{tcolorbox}[breakable, size=fbox, boxrule=1pt, pad at break*=1mm,colback=cellbackground, colframe=cellborder]
\prompt{In}{incolor}{51}{\boxspacing}
\begin{Verbatim}[commandchars=\\\{\}]
\PY{n}{matrix} \PY{o}{=} \PY{n}{B}\PY{p}{(}\PY{n}{epses}\PY{p}{[}\PY{l+m+mi}{3}\PY{p}{]}\PY{p}{)}
\PY{n}{inverse\PYZus{}matrix} \PY{o}{=} \PY{n}{sla}\PY{o}{.}\PY{n}{inv}\PY{p}{(}\PY{n}{matrix}\PY{p}{)}
\PY{n}{test}\PY{p}{(}\PY{n}{matrix}\PY{p}{,} \PY{n}{inverse\PYZus{}matrix}\PY{p}{)}
\end{Verbatim}
\end{tcolorbox}

    \hypertarget{eps-0.000000000000001}{%
\subsection{eps = 0.000000000000001}\label{eps-0.000000000000001}}

Рассмотрим \(B_{eps}\) при \(eps = 0.000000000000001\). В данном случае,
алгоритм выдает сообщение, о том что матрица является вырожденной.

    \begin{tcolorbox}[breakable, size=fbox, boxrule=1pt, pad at break*=1mm,colback=cellbackground, colframe=cellborder]
\prompt{In}{incolor}{50}{\boxspacing}
\begin{Verbatim}[commandchars=\\\{\}]
\PY{n}{matrix} \PY{o}{=} \PY{n}{B}\PY{p}{(}\PY{n}{epses}\PY{p}{[}\PY{l+m+mi}{4}\PY{p}{]}\PY{p}{)}
\PY{n}{inverse\PYZus{}matrix} \PY{o}{=} \PY{n}{inverse}\PY{p}{(}\PY{n}{matrix}\PY{p}{)}
\PY{n}{test}\PY{p}{(}\PY{n}{matrix}\PY{p}{,} \PY{n}{inverse\PYZus{}matrix}\PY{p}{)}
\end{Verbatim}
\end{tcolorbox}

    \begin{Verbatim}[commandchars=\\\{\}]
<ipython-input-9-089af5996201>:33: LinAlgWarning: Singular matrix
  warnings.warn("Singular matrix", sla.LinAlgWarning)
    \end{Verbatim}

    При этом, точность встроенного метод \texttt{inv} падает, поэтому
условие проверки в данном тестовом случае не соблюдено.

    \begin{tcolorbox}[breakable, size=fbox, boxrule=1pt, pad at break*=1mm,colback=cellbackground, colframe=cellborder]
\prompt{In}{incolor}{55}{\boxspacing}
\begin{Verbatim}[commandchars=\\\{\}]
\PY{n}{matrix} \PY{o}{=} \PY{n}{B}\PY{p}{(}\PY{n}{epses}\PY{p}{[}\PY{l+m+mi}{4}\PY{p}{]}\PY{p}{)}
\PY{n}{inverse\PYZus{}matrix} \PY{o}{=} \PY{n}{sla}\PY{o}{.}\PY{n}{inv}\PY{p}{(}\PY{n}{matrix}\PY{p}{)}
\PY{n}{test}\PY{p}{(}\PY{n}{matrix}\PY{p}{,} \PY{n}{inverse\PYZus{}matrix}\PY{p}{)}
\end{Verbatim}
\end{tcolorbox}

    \begin{Verbatim}[commandchars=\\\{\}, frame=single, framerule=2mm, rulecolor=\color{outerrorbackground}]
\textcolor{ansi-red}{---------------------------------------------------------------------------}
\textcolor{ansi-red}{AssertionError}                            Traceback (most recent call last)
\textcolor{ansi-green}{<ipython-input-55-63b1598fd2e0>} in \textcolor{ansi-cyan}{<module>}
\textcolor{ansi-green-intense}{\textbf{      1}} matrix \textcolor{ansi-blue}{=} B\textcolor{ansi-blue}{(}epses\textcolor{ansi-blue}{[}\textcolor{ansi-cyan}{4}\textcolor{ansi-blue}{]}\textcolor{ansi-blue}{)}
\textcolor{ansi-green-intense}{\textbf{      2}} inverse\_matrix \textcolor{ansi-blue}{=} sla\textcolor{ansi-blue}{.}inv\textcolor{ansi-blue}{(}matrix\textcolor{ansi-blue}{)}
\textcolor{ansi-green}{----> 3}\textcolor{ansi-red}{ }test\textcolor{ansi-blue}{(}matrix\textcolor{ansi-blue}{,} inverse\_matrix\textcolor{ansi-blue}{)}

\textcolor{ansi-green}{<ipython-input-33-327f4077c160>} in \textcolor{ansi-cyan}{test}\textcolor{ansi-blue}{(matrix, inverse\_matrix, eps)}
\textcolor{ansi-green-intense}{\textbf{      1}} \textcolor{ansi-green}{def} test\textcolor{ansi-blue}{(}matrix\textcolor{ansi-blue}{:} np\textcolor{ansi-blue}{.}array\textcolor{ansi-blue}{,} inverse\_matrix\textcolor{ansi-blue}{:} np\textcolor{ansi-blue}{.}array\textcolor{ansi-blue}{,} eps \textcolor{ansi-blue}{=} \textcolor{ansi-cyan}{1e-5}\textcolor{ansi-blue}{)} \textcolor{ansi-blue}{->} \textcolor{ansi-green}{None}\textcolor{ansi-blue}{:}
\textcolor{ansi-green-intense}{\textbf{      2}}     \textcolor{ansi-green}{if} inverse\_matrix \textcolor{ansi-green}{is} \textcolor{ansi-green}{not} \textcolor{ansi-green}{None}\textcolor{ansi-blue}{:}
\textcolor{ansi-green}{----> 3}\textcolor{ansi-red}{         }\textcolor{ansi-green}{assert} sla\textcolor{ansi-blue}{.}norm\textcolor{ansi-blue}{(}matrix \textcolor{ansi-blue}{@} inverse\_matrix \textcolor{ansi-blue}{-} np\textcolor{ansi-blue}{.}eye\textcolor{ansi-blue}{(}matrix\textcolor{ansi-blue}{.}shape\textcolor{ansi-blue}{[}\textcolor{ansi-cyan}{0}\textcolor{ansi-blue}{]}\textcolor{ansi-blue}{)}\textcolor{ansi-blue}{)} \textcolor{ansi-blue}{<} eps

\textcolor{ansi-red}{AssertionError}: 
    \end{Verbatim}

    \hypertarget{ux442ux435ux441ux442ux438ux440ux43eux432ux430ux43dux438ux435-ux430ux43bux433ux43eux440ux438ux442ux43cux430-ux43dux430-ux43cux430ux442ux440ux438ux446ux435-c_14-ux434ux43bux44f-ux43eux43fux440ux435ux434ux435ux43bux435ux43dux438ux44f-ux441ux43eux43eux442ux432ux435ux442ux441ux442ux432ux438ux44f-ux440ux435ux430ux43bux438ux437ux43eux432ux430ux43dux43dux43eux433ux43e-ux430ux43bux433ux43eux440ux438ux442ux43cux430-ux437ux430ux43bux43eux436ux435ux43dux43dux43eux43cux443-ux43cux435ux442ux43eux434ux443}{%
\section{\texorpdfstring{Тестирование алгоритма на матрице \(C_{14}\)
для определения соответствия реализованного алгоритма заложенному
методу}{Тестирование алгоритма на матрице C\_\{14\} для определения соответствия реализованного алгоритма заложенному методу}}\label{ux442ux435ux441ux442ux438ux440ux43eux432ux430ux43dux438ux435-ux430ux43bux433ux43eux440ux438ux442ux43cux430-ux43dux430-ux43cux430ux442ux440ux438ux446ux435-c_14-ux434ux43bux44f-ux43eux43fux440ux435ux434ux435ux43bux435ux43dux438ux44f-ux441ux43eux43eux442ux432ux435ux442ux441ux442ux432ux438ux44f-ux440ux435ux430ux43bux438ux437ux43eux432ux430ux43dux43dux43eux433ux43e-ux430ux43bux433ux43eux440ux438ux442ux43cux430-ux437ux430ux43bux43eux436ux435ux43dux43dux43eux43cux443-ux43cux435ux442ux43eux434ux443}}

    Рассмотрим ход работы метода для матрицы вида: \[C_{14} = 
\begin{pmatrix} 
-1 && -1 && -1 && -1 \\ 
0 && 0 && 2 && 6 \\ 
1 && 2 && 0 && 0 \\ 
1 && -2 && 1 && 1
\end{pmatrix}
\]

\textbf{Итерация} \(i = 0\)

Положим: - \(C^0 = E\) - матрица, которая после замены столбцов на
столбцы \(C_{14}\) (после всех итераций алгоритма) должна стать равной
\(C_{14}\) - \(B^0 = E\) - матрица, которая после всех итераций
алгоритма должна стать равной \(C_{14}^{-1}\) - \(J^0 = \{1, 2, 3, 4\}\)
- множество номеров неиспользованных (невставленных) столбцев исходной
матрицы \(C_{14}\)

При этом, столбы матрицы \(C_{14}\) обозначим как \(c_1\), \(c_2\),
\(c_3\) ,\(c_4\)

Шаг 1. Найдем ненулевое \(\alpha_j\), где \(j\) итерируется по номерам
неиспользованных столбцев из множества \(J^0\): -
\(\alpha_1 = e_1^TB^0c_1 = (1, 0, 0, 0) \begin{pmatrix} 1 && 0 && 0 && 0 \\ 0 && 1 && 0 && 0 \\ 0 && 0 && 1 && 0 \\ 0 && 0 && 0 && 1 \end{pmatrix} \begin{pmatrix}-1 \\ 0 \\ 1 \\ 1\end{pmatrix} = 1\)

Поскольку \(\alpha_1 \ne 0\), то для вставки выберем 1 столбец.

Шаг 2. Обновим \(C^1, B^1, J^1\): - \(C^1 = (c_1, e_2, e_3, e_4)\) -
\(J^1 = J^0 \backslash \{1\} = \{2, 3, 4 \}\) -
\(B^1 = D(1, B^0c_1)B^0 = \begin{pmatrix} -1 && 0 && 0 && 0 \\ 0 && 1 && 0 && 0 \\ 1 && 0 && 1 && 0 \\ 1 && 0 && 0 && 1 \end{pmatrix} \begin{pmatrix} 1 && 0 && 0 && 0 \\ 0 && 1 && 0 && 0 \\ 0 && 0 && 1 && 0 \\ 0 && 0 && 0 && 1 \end{pmatrix} = \begin{pmatrix} -1 && 0 && 0 && 0 \\ 0 && 1 && 0 && 0 \\ 1 && 0 && 1 && 0 \\ 1 && 0 && 0 && 1 \end{pmatrix}\)

Так как номер итерации \(i + 1 < 4\), перейдем на следующую итерацию
\(i = 1\).

\textbf{Итерация} \(i = 1\)

Шаг 1. Найдем ненулевое \(\alpha_j\), где \(j\) итерируется по номерам
неиспользованных столбцев из множества \(J^1\): -
\(\alpha_2 = e_2^TB^1c_2 = (0, 1, 0, 0) \begin{pmatrix} -1 && 0 && 0 && 0 \\ 0 && 1 && 0 && 0 \\ 1 && 0 && 1 && 0 \\ 1 && 0 && 0 && 1 \end{pmatrix} \begin{pmatrix}-1 \\ 0 \\ 2 \\ -2\end{pmatrix} = 0\)

Так как \(\alpha_2 = 0\), то продолжим искать подходящий номер столбца

\begin{itemize}
\tightlist
\item
  \(\alpha_3 = e_2^TB^1c_3 = (0, 1, 0, 0) \begin{pmatrix} -1 && 0 && 0 && 0 \\ 0 && 1 && 0 && 0 \\ 1 && 0 && 1 && 0 \\ 1 && 0 && 0 && 1 \end{pmatrix} \begin{pmatrix}-1 \\ 2 \\ 0 \\ 1\end{pmatrix} = 2\)
\end{itemize}

Поскольку \(\alpha_3 \ne 0\), то для вставки выберем 3 столбец.

Шаг 2. Обновим \(C^2, B^2, J^2\): - \(C^2 = (c_1, c_3, e_3, e_4)\) -
\(J^2 = J^1 \backslash \{3\} = \{2, 4 \}\) -
\(B^2 = D(2, B^1c_3)B^1 = \begin{pmatrix} -1 && -0.5 && 0 && 0 \\ 0 && 0.5 && 0 && 0 \\ 1 && 0.5 && 1 && 0 \\ 1 && 0 && 0 && 1 \end{pmatrix}\)

Так как номер итерации \(i + 1 < 4\), перейдем на следующую итерацию
\(i = 2\).

\textbf{Итерация} \(i = 2\)

Шаг 1. Найдем ненулевое \(\alpha_j\), где \(j\) итерируется по номерам
неиспользованных столбцев из множества \(J^2\): -
\(\alpha_2 = e_3^TB^2c_2 = (0, 0, 1, 0) \begin{pmatrix} -1 && -0.5 && 0 && 0 \\ 0 && 0.5 && 0 && 0 \\ 1 && 0.5 && 1 && 0 \\ 1 && 0 && 0 && 1 \end{pmatrix} \begin{pmatrix}-1 \\ 0 \\ 2 \\ -2\end{pmatrix} = 2\)

Так как \(\alpha_2 \ne 0\), то для вставки выберем 2 столбец.

Шаг 2. Обновим \(C^3, B^3, J^3\): - \(C^3 = (c_1, c_3, c_2, e_4)\) -
\(J^3 = J^2 \backslash \{2\} = \{4 \}\) -
\(B^3 = D(3, B^2c_2)B^2 = \begin{pmatrix} -2 && -1 && -1 && 0 \\ 0 && 0.5 && 0 && 0 \\ 1 && 0.5 && 1 && 0 \\ 4 && 1.5 && 3 && 1 \end{pmatrix}\)

Так как номер итерации \(i + 1 < 4\), перейдем на следующую итерацию
\(i = 3\).

\textbf{Итерация} \(i = 3\)

Шаг 1. Найдем ненулевое \(\alpha_j\), где \(j\) итерируется по номерам
неиспользованных столбцев из множества \(J^3\): -
\(\alpha_4 = e_4^TB^3c_4 = (0, 0, 0, 1) \begin{pmatrix} -2 && -1 && -1 && 0 \\ 0 && 0.5 && 0 && 0 \\ 1 && 0.5 && 1 && 0 \\ 4 && 1.5 && 3 && 1 \end{pmatrix} \begin{pmatrix}-1 \\ 6 \\ 0 \\ 1\end{pmatrix} = 1\)

Так как \(\alpha_4 \ne 0\), то для вставки выберем 4 столбец.

Шаг 2. Обновим \(C^4, B^4, J^4\): - \(C^4 = (c_1, c_3, c_2, c_4)\) -
\(J^4 = J^3 \backslash \{4\} = \{ \}\) -
\(B^4 = D(4, B^3c_4)B^3 = \begin{pmatrix} -6.66 && -5.55 && 1 && 6.67 \\ -2.00 && -2.50 && -1.5 && -5.00 \\ 3.33 && 2.78 && 5.55 && -3.33 \\ 6.66 && 2.50 && 5.00 && 1.67 \end{pmatrix}\)

Так как номер итерации \(i + 1 = 4\), работа метода окончена. Искомая
\(C_{14}^{-1}\) матрица получается перестановкой строк матрицы \(B^4\)

    Протестируем реализованный алгоритм на матрице \(C_{14}\), при этом
делая вывод текущей итерации \(i\), множества \(J_i\) и матрицы \(B_i\).
Как видно, реализованный алгоритм показывает полное соответствие
заложенному в него методу.

    \begin{tcolorbox}[breakable, size=fbox, boxrule=1pt, pad at break*=1mm,colback=cellbackground, colframe=cellborder]
\prompt{In}{incolor}{73}{\boxspacing}
\begin{Verbatim}[commandchars=\\\{\}]
\PY{n}{matrix} \PY{o}{=} \PY{n}{np}\PY{o}{.}\PY{n}{array}\PY{p}{(}\PY{p}{[}\PY{p}{[}\PY{o}{\PYZhy{}}\PY{l+m+mi}{1}\PY{p}{,} \PY{o}{\PYZhy{}}\PY{l+m+mi}{1}\PY{p}{,} \PY{o}{\PYZhy{}}\PY{l+m+mi}{1}\PY{p}{,} \PY{o}{\PYZhy{}}\PY{l+m+mi}{1}\PY{p}{]}\PY{p}{,} \PY{p}{[}\PY{l+m+mi}{0}\PY{p}{,} \PY{l+m+mi}{0}\PY{p}{,} \PY{l+m+mi}{2}\PY{p}{,} \PY{l+m+mi}{6}\PY{p}{]}\PY{p}{,} \PY{p}{[}\PY{l+m+mi}{1}\PY{p}{,} \PY{l+m+mi}{2}\PY{p}{,} \PY{l+m+mi}{0}\PY{p}{,} \PY{l+m+mi}{0}\PY{p}{]}\PY{p}{,} \PY{p}{[}\PY{l+m+mi}{1}\PY{p}{,} \PY{o}{\PYZhy{}}\PY{l+m+mi}{2}\PY{p}{,} \PY{l+m+mi}{1}\PY{p}{,} \PY{l+m+mi}{1}\PY{p}{]}\PY{p}{]}\PY{p}{)}
\PY{n}{inverse\PYZus{}matrix} \PY{o}{=} \PY{n}{inverse}\PY{p}{(}\PY{n}{matrix}\PY{p}{,} \PY{n}{verbose}\PY{o}{=}\PY{k+kc}{True}\PY{p}{)}
\PY{n}{test}\PY{p}{(}\PY{n}{matrix}\PY{p}{,} \PY{n}{inverse\PYZus{}matrix}\PY{p}{)}
\end{Verbatim}
\end{tcolorbox}

    \begin{Verbatim}[commandchars=\\\{\}]
Initial unused indexes set \{1, 2, 3, 4\}
Initial matrix
 [[1. 0. 0. 0.]
 [0. 1. 0. 0.]
 [0. 0. 1. 0.]
 [0. 0. 0. 1.]]
Current iteration no. 1
Current indexes set \{2, 3, 4\}
Current inverse matrix
 [[-1.  0.  0.  0.]
 [ 0.  1.  0.  0.]
 [ 1.  0.  1.  0.]
 [ 1.  0.  0.  1.]]
Current iteration no. 2
Current indexes set \{2, 4\}
Current inverse matrix
 [[-1.  -0.5  0.   0. ]
 [ 0.   0.5  0.   0. ]
 [ 1.   0.5  1.   0. ]
 [ 1.   0.   0.   1. ]]
Current iteration no. 3
Current indexes set \{4\}
Current inverse matrix
 [[-2.  -1.  -1.   0. ]
 [ 0.   0.5  0.   0. ]
 [ 1.   0.5  1.   0. ]
 [ 4.   1.5  3.   1. ]]
Current iteration no. 4
Current indexes set \{\}
Current inverse matrix
 [[ 6.66666667e-01 -5.55111512e-17  1.00000000e+00  6.66666667e-01]
 [-2.00000000e+00 -2.50000000e-01 -1.50000000e+00 -5.00000000e-01]
 [-3.33333333e-01  2.77555756e-17  5.55111512e-17 -3.33333333e-01]
 [ 6.66666667e-01  2.50000000e-01  5.00000000e-01  1.66666667e-01]]
    \end{Verbatim}


    % Add a bibliography block to the postdoc
    
    
    
\end{document}
